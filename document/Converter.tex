\documentclass[hidelinks,12pt]{article}
\linespread{1.3}
\usepackage{hyperref}
\usepackage{enumitem}
%\usepackage{enumerate}
\usepackage{changepage,lipsum,titlesec, longtable}
\usepackage{cite}
\usepackage{comment, xcolor}
\usepackage[pdftex]{graphicx}
  \graphicspath{{images/}, {images/stat/}}
  \DeclareGraphicsExtensions{.pdf,.jpeg,.png, .jpg}
\usepackage[cmex10]{amsmath}
\usepackage{tikz}
\usepackage{array} 
\usepackage{subfigure} 
\newcommand{\grey}[1]{\textcolor{black!30}{#1}}
\newcommand{\red}[1]{\textcolor{red!50}{#1}}
\newcommand{\fref}[1]{Figure \ref{#1}}
\newcommand{\tref}[1]{Table \ref{#1}}
\newcommand{\eref}[1]{Equation~\ref{#1}}
\newcommand{\cref}[1]{Chapter~\ref{#1}}
\newcommand{\sref}[1]{Section~\ref{#1}}
\newcommand{\aref}[1]{Appendix~\ref{#1}}

\renewcommand{\labelenumii}{\theenumii}
\renewcommand{\theenumii}{\theenumi.\\arabic{enumii}.}

\oddsidemargin0cm
\topmargin-2cm %I recommend adding these three lines to increase the
\textwidth16.5cm %amount of usable space on the page (and save trees)
\textheight23.5cm

\makeatletter
\renewcommand\paragraph{\@startsection{paragraph}{4}{\z@}%
            {-2.5ex\@plus -1ex \@minus -.25ex}%
            {1.25ex \@plus .25ex}%
            {\normalfont\normalsize\bfseries}}
\makeatother
\setcounter{secnumdepth}{4} % how many sectioning levels to assign numbers to
\setcounter{tocdepth}{4}    % how many sectioning levels to show in ToC


\begin{document}
\title{EnergyPlus Water-to-air Heat Pump\\
       \large Ambient Water Loop System Modeling}
\maketitle
\tableofcontents
\newpage
\section{Definitions}\label{sec:def}
\begin{description}
\item[heat pump] An equipment that ``extracts heat from a source and
  transfers it to a sink at a higher temperature.'' ~\cite{ASHRAE2012}
  Heat sources of a heat pump could be ``water, solar energy, the air
  and internal building heat''~\cite{ASHRAE2012} Commonly used water
  sources include ground water, surface water, cooling tower, a closed
  water-loop in a building or a community, or gray water.
\item[water-to-air heat pump] heat pumps that use water as the heat
  source and sink and use air to deliver thermal energy to conditioned
  space~\cite{ASHRAE2012}.
\item[water-loop heat pump (WLHP)] system: there are one or more heat
  pumps in each zone. The heat pumps are connected to a shared pipe
  system with one supply pipe and one return pipe. The water loop has
  a cooling tower to reject heat, a heat source, usually a boiler, to
  add heat and two pumps (one backup) for circulation. The units
  absorbs heat from the loop if it is in heating mode and dumps heat
  to the loop if it is in cooling mode. If the water loop is connected
  to the ground (ground-coupled), then no additional boilers or
  cooling towers are needed. The system would be beneficial for
  buildings or building groups with a coincident heating and cooling
  demand. One common application of WLHP is in office buildings where
  in the heating season, internal gains from core zones can be
  redirected to perimeter zones~\cite{ASHRAE2012}.
  
  The control of WLHP system: heat pump units are controlled by zone
  thermostat. The boiler is controlled so that the boiler output water
  temperature is at 60F (50F?) The cooling tower is controlled so that
  its output temperature is 90F~\cite{ASHRAE2012}.
\end{description}
\section{General Introduction}\label{sec:intro}
The work at this stage is to model a water ambient loop system that
connects multiple zones of a group of buildings that could have
different thermal demand. Before this, a single-building water-loop
heat pump system (WLHP) needs to be implemented and tested. 

According to ASHRAE project RP620[cite], the computer models for WLHP
system has large variations.
\newpage
\bibliographystyle{plain}
\bibliography{myCitation}
\end{document}