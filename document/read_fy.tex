\documentclass[12pt]{article}
\linespread{1.3}
\usepackage{hyperref}
\usepackage{enumitem}
%\usepackage{enumerate}
\usepackage{changepage,lipsum,titlesec, longtable}
\usepackage{cite}
\usepackage{comment, xcolor}
\usepackage[pdftex]{graphicx}
  \graphicspath{{images/}, {images/stat/}}
  \DeclareGraphicsExtensions{.pdf,.jpeg,.png, .jpg}
\usepackage[cmex10]{amsmath}
\usepackage{tikz}
\usepackage{array} 
\usepackage{subfigure} 
\newcommand{\grey}[1]{\textcolor{black!30}{#1}}
\newcommand{\red}[1]{\textcolor{red!50}{#1}}
\newcommand{\question}[1]{\textcolor{magenta}{\textbf{Question: } {#1}}}
\newcommand{\fref}[1]{Figure~\ref{#1}}
\newcommand{\tref}[1]{Table~\ref{#1}}
\newcommand{\eref}[1]{Equation~\ref{#1}}
\newcommand{\cref}[1]{Chapter~\ref{#1}}
\newcommand{\sref}[1]{Section~\ref{#1}}
\newcommand{\aref}[1]{Appendix~\ref{#1}}
\newcommand{\note}[0]{\textbf{Note: }}

\renewcommand{\labelenumii}{\theenumii}
\renewcommand{\theenumii}{\theenumi.\\arabic{enumii}.}

\oddsidemargin0cm
\topmargin-2cm %I recommend adding these three lines to increase the
\textwidth16.5cm %amount of usable space on the page (and save trees)
\textheight23.5cm

\makeatletter
\renewcommand\paragraph{\@startsection{paragraph}{4}{\z@}%
            {-2.5ex\@plus -1ex \@minus -.25ex}%
            {1.25ex \@plus .25ex}%
            {\normalfont\normalsize\bfseries}}
\makeatother
\setcounter{secnumdepth}{4} % how many sectioning levels to assign numbers to
\setcounter{tocdepth}{4}    % how many sectioning levels to show in ToC

% draw diagram
\usetikzlibrary{shapes.geometric, arrows}
\tikzstyle{data} = [font=\scriptsize, rectangle, rounded corners, minimum width=1.5cm, minimum height=1cm,align=left, draw=black, fill=black!30]
\tikzstyle{database} = [font=\scriptsize, rectangle, rounded corners, minimum width=3cm, minimum height=1cm,align=left, draw=black, fill=green!30]
\tikzstyle{query} = [font=\scriptsize,trapezium, trapezium left angle=70, trapezium right angle=110, minimum width=0.5cm, minimum height=0.5cm, text centered, draw=black, fill=blue!30]
\tikzstyle{process} = [font=\scriptsize,rectangle, minimum width=3cm, minimum height=1cm, text centered, draw=black, fill=orange!30]
\tikzstyle{spliter} = [font=\scriptsize,diamond, minimum width=2cm, minimum height=1cm, text centered, draw=black, fill=green!30]
\tikzstyle{decision} = [font=\scriptsize,diamond, minimum width=3cm, minimum height=1cm, text centered, draw=black, fill=green!30]
\tikzstyle{arrow} = [thick,->,>=stealth]
\tikzstyle{bi-arrow} = [thick,->,>=stealth]

\begin{document}
\title{Data processing for PM data\\
       \large GSA project}
\maketitle
\tableofcontents
\newpage
\section{Introduction}\label{sec:intro}
The document records the process of plotting the new FY data 

\section{Checks}
\subsection{Sheets}
sheet names:
\makeatletter
\def\verbatim@font{\linespread{1}\small\ttfamily}
\begin{verbatim}
# FY 13
[u'R1', u'R2', u'R3', u'R4', u'R5', u'R6', u'R7', u'R8', u'R9', u'R10', u'R11']
# FY 14
[u'r1 fy14', u'r2 fy14', u'r3 fy14', u'r4 fy14', u'r5 fy14', u'r6 fy14', u'r7 fy14', u'r8 fy14', u'r9 fy14', u'r10 fy14', u'r11 fy14', u'Sheet2']
# FY 15
[u'R1 FY15', u'R2 FY15', u'R3 FY15', u'R4 FY15', u'R5 FY15', u'R6 FY15', u'R7 FY15', u'R8 FY15', u'R9 FY15', u'R10 FY15', u'R11 FY15']
\end{verbatim}
The names of sheets are not standard but the order of region is, so use index of sheets when reading.
For faster processing, split excel files to single sheet csv files
\subsection{Number of building}
\begin{table}[h!]
\centering
\caption{Number of buildings in each file}
\label{tab:num_bd}
\begin{tabular}{llll}
  \hline
Region&FY13 & FY14 & FY15\\
  \hline
1    & 76   & 74   & 74  \\
2    & 167  & 165  & 154 \\
3    & 80   & 79   & 78  \\
4    & 127  & 126  & 130 \\
5    & 160  & 150  & 146 \\
6    & 43   & 40   & 40  \\
7    & 133  & 131  & 129 \\
8    & 57   & 56   & 56  \\
9    & 75   & 74   & 72  \\
10   & 62   & 59   & 59  \\
11   & 119  & 130  & 117 \\
  \hline
\end{tabular}
\end{table} 

\subsection{Common buildings}
There are 1012 buildings that has records for all three years.\\
There are 1141 buildings covered in one of the three files.
\begin{table}[h!]
\centering
\caption{Building record and its year}
\label{my-label}
\begin{tabular}{llll}
  \hline
Building Number & 2013 & 2014 & 2015 \\
  \hline
NJ4666ZZ        & 0    & 0    & 1    \\
NY7182ZZ        & 1    & 1    & 1    \\
OR0053ZZ        & 1    & 1    & 1    \\
IA0121ZZ        & 1    & 1    & 1    \\
OH2433ZZ        & 1    & 1    & 1    \\
IL2540ZZ        & 1    & 1    & 1    \\
AZ0000WW        & 1    & 1    & 1    \\
MA0050ZZ        & 1    & 1    & 1    \\
TX2747ZZ        & 1    & 1    & 1    \\
ME0009ZZ        & 1    & 0    & 0    \\
FL3323ZZ        & 0    & 1    & 1    \\
  \hline
\end{tabular}
\end{table}

note: ``1'' denotes the building has records in the corresponding
year of the column, ``0'' denotes there are no records for this
building in that year

\section{Splitting buildings to single csv files}
\section{For each single building file calculate eui}
Check the buildings with monthly energy records:
\begin{verbatim}
NY7288ZZ (region 2) in 2013 has records for only Apr. to Dec.
VI3977ZZ (region 2) in 2013 has records for only Oct. to Dec.
VI3987ZZ (region 2) in 2013 has records for only Jul. to Dec.
\end{verbatim}

\section{Calculate annual eui}
Note, for the three buildings above, there will be an under-estimate.
\section{Join eui with program information}

\newpage
\bibliographystyle{plain} \bibliography{myCitation}
\end{document}
