\documentclass[12pt]{article}
\linespread{1.3}
\usepackage{hyperref}
\usepackage{enumitem}
%\usepackage{enumerate}
\usepackage{changepage,lipsum,titlesec, longtable}
\usepackage{cite}
\usepackage{comment, xcolor}
\usepackage[pdftex]{graphicx}
  \graphicspath{{images/}, {images/stat/}}
  \DeclareGraphicsExtensions{.pdf,.jpeg,.png, .jpg}
\usepackage[cmex10]{amsmath}
\usepackage{tikz}
\usepackage{array} 
\usepackage{subfigure} 
\newcommand{\grey}[1]{\textcolor{black!30}{#1}}
\newcommand{\red}[1]{\textcolor{red!50}{#1}}
\newcommand{\question}[1]{\textcolor{magenta}{\textbf{Question: } {#1}}}
\newcommand{\fref}[1]{Figure~\ref{#1}}
\newcommand{\tref}[1]{Table~\ref{#1}}
\newcommand{\eref}[1]{Equation~\ref{#1}}
\newcommand{\cref}[1]{Chapter~\ref{#1}}
\newcommand{\sref}[1]{Section~\ref{#1}}
\newcommand{\aref}[1]{Appendix~\ref{#1}}
\newcommand{\note}[0]{\textbf{Note: }}

\renewcommand{\labelenumii}{\theenumii}
\renewcommand{\theenumii}{\theenumi.\\arabic{enumii}.}

\oddsidemargin0cm
\topmargin-2cm %I recommend adding these three lines to increase the
\textwidth16.5cm %amount of usable space on the page (and save trees)
\textheight23.5cm

\makeatletter
\renewcommand\paragraph{\@startsection{paragraph}{4}{\z@}%
            {-2.5ex\@plus -1ex \@minus -.25ex}%
            {1.25ex \@plus .25ex}%
            {\normalfont\normalsize\bfseries}}
\makeatother
\setcounter{secnumdepth}{4} % how many sectioning levels to assign numbers to
\setcounter{tocdepth}{4}    % how many sectioning levels to show in ToC

% draw diagram
\usetikzlibrary{shapes.geometric, arrows}
\tikzstyle{data} = [font=\scriptsize, rectangle, rounded corners, minimum width=1.5cm, minimum height=1cm,align=left, draw=black, fill=black!30]
\tikzstyle{database} = [font=\scriptsize, rectangle, rounded corners, minimum width=3cm, minimum height=1cm,align=left, draw=black, fill=green!30]
\tikzstyle{query} = [font=\scriptsize,trapezium, trapezium left angle=70, trapezium right angle=110, minimum width=0.5cm, minimum height=0.5cm, text centered, draw=black, fill=blue!30]
\tikzstyle{process} = [font=\scriptsize,rectangle, minimum width=3cm, minimum height=1cm, text centered, draw=black, fill=orange!30]
\tikzstyle{spliter} = [font=\scriptsize,diamond, minimum width=2cm, minimum height=1cm, text centered, draw=black, fill=green!30]
\tikzstyle{decision} = [font=\scriptsize,diamond, minimum width=3cm, minimum height=1cm, text centered, draw=black, fill=green!30]
\tikzstyle{arrow} = [thick,->,>=stealth]
\tikzstyle{bi-arrow} = [thick,->,>=stealth]


\begin{document}
\title{Excel converter for Jihyun\\
       \large GSA project}
\maketitle
\tableofcontents
\newpage
\section{Introduction}\label{sec:intro}
The document records the implementation of the data converter from
excel to csv for Jihyn. 

\section{Definition of the mapping from source to target format}
Via the discussion in the first meeting, the mapping from source PM
table (containing energy consumption information) to the target EUAS
template is depicted in \tref{tab:mapField}:
\begin{table}[h!]
  \scriptsize
\centering
\caption{Excel merging script field mapping table}
\label{tab:mapField}
\begin{tabular}{p{3cm}p{3cm}|p{3cm}|p{5cm}}
Source                 &                              & target                      & Condition                   \\
sheet name             & field name                   &                             &                             \\
  \hline
  \hline
Properties             & State/Province               & State                       &                             \\
  \hline
Properties             & Gross Floor Area             & GSF                         &                             \\
  \hline
Properties             & Year Built                   & Year Built                  &                             \\
  \hline
Properties             & Postal Code                  & Postal Code (first 5 digit) &                           \\ 
  \hline
Property Use & Self-Selected Primary Function         & Use Type                    \\ 
  \hline
  \hline
Meter Consumption Data & Property Name                & Building ID                 &                             \\
  \hline
Meter Consumption Data & End Date (month)             & Month                       &                             \\
  \hline
Meter Consumption Data & End Date (year)             & Year                        &                             \\
  \hline
Meter Consumption Data & (Usage/Quantity, Meter Type) & Elec Amt                    & if Meter Type = Electric - Grid\\
  \cline{3-4}
                       &                              & Gas Amt                     & if Meter Type = Natural Gas\\
  \cline{3-4}
                       &                              & Oil Amt                     & if Meter Type = Fuel Oil (No. 2)\\
  \cline{3-4}
                       &                              & Water Amt                   & if Meter Type = Potable: Mixed Indoor/Outdoor\\
  \hline
Meter Consumption Data & (Usage/Quantity, Meter Type) & Elec Cost                    & if Meter Type = Electric - Grid\\
  \cline{3-4}
                       &                              & Gas Cost                     & if Meter Type = Natural Gas\\
  \cline{3-4}
                       &                              & Oil Cost                     & if Meter Type = Fuel Oil (No. 2)\\
  \cline{3-4}
                       &                              & Water Cost                   & if Meter Type = Potable: Mixed Indoor/Outdoor\\
  \hline
Meter Consumption Data & Portfolio Manager ID         & Portfolio Manager ID        &                             \\
  \hline
Meter Consumption Data & Portfolio Manager Meter ID   & Portfolio Manager Meter ID  &                             \\
  \hline
  \hline
\end{tabular}
\end{table}
As described in following sections, this template design of creating
several columns (Usage/Quantity, Meter ID, unit, Cost (\$))for each
new energy usage type have several draw backs:
\begin{itemize}
\item The large number of different energy usage types (19 in total)
  will result in large number of columns
\item The large variety in the number of energy record for each types
  of energy source will result in a large number of waste space in the
  table: for example, there are only 12 records for 'Electric - Wind'
  but 111065 records for 'Electric - Grid'. For the EUAS template with
  different columns for different energy source, there will be $111065
  - 12 = 111053$ empty cells in the column of 'Electric - Wind'
\item If in the future, the new type of energy usage is included, the
  structure of the table will change (the number of columns), this
  requires reprocessing the whole table again. If we maintain the way
  of keeping all energy consumption stacked on top of each other (the
  way in PM table), we can either append records of the new resource
  to the end or save it to another table without affecting the already
  tidied data.
\end{itemize}

Thus I propose to maintain the structure of the PM in recording energy
usage information. Hence the mapping of fields from source to target
table would be as the following:

\begin{table}[h!]
  \scriptsize
\centering
\caption{Excel merging script field mapping table}
\label{tab:mapFieldNew}
\begin{tabular}{c | c | c}%{p{3cm}p{3cm}|p{3cm}}
  \hline
Source                 &                              & target                                         \\
sheet name             & field name                   &                                                          \\
  \hline
  \hline
Properties             & State/Province               & State                                                    \\
  \hline
Properties             & Country               & Country\\
  \hline
Properties             & Gross Floor Area             & GSF                                                      \\
  \hline
Properties             & Year Built                   & Year Built                                               \\
  \hline
Properties             & Postal Code                  & Postal Code (first 5 digit)                            \\ 
  \hline
Property Use & Self-Selected Primary Function         & Use Type                    \\ 
  \hline
  \hline
Meter Consumption Data & Property Name                & Building ID                                              \\
  \hline
Meter Consumption Data & End Date (month)             & Month                                                    \\
  \hline
Meter Consumption Data & End Date (year)             & Year                                                     \\
  \hline
Meter Consumption Data & (Usage/Quantity, Meter Type) & Usage/Quantity\\
  \hline
Meter Consumption Data & (Usage/Quantity, Meter Type) & Cost (\$)                    \\
  \hline
Meter Consumption Data & Portfolio Manager ID         & Portfolio Manager ID                                     \\
  \hline
Meter Consumption Data & Portfolio Manager Meter ID   & Portfolio Manager Meter ID                               \\
  \hline
  \hline
\end{tabular}
\end{table}
\subsection{Input data quality check - Energy Information}
\subsubsection{Number of missing value}
From the output of the script, one can see there are missing data in
the following fields:
\begin{itemize}
\item End Date: 39 missing data.
\item ``Cost\$'': 12942 missing data.
\end{itemize}

For ``End Date'', we will discard these records with, for ``Cost
(\$)'', we'll first mark the missing data with ``-1'', and discard it
when doing cost related analysis
\makeatletter
\def\verbatim@font{\linespread{1}\small\ttfamily}
\begin{verbatim}
checking numer of missing values for columns
## ------------------------------------------##
Portfolio Manager ID
non_Null    344509
dtype: int64
## ------------------------------------------##
Portfolio Manager Meter ID
non_Null    344509
dtype: int64
## ------------------------------------------##
Meter Type
non_Null    344509
dtype: int64
## ------------------------------------------##
End Date
non_Null    344470
Null            39
dtype: int64
## ------------------------------------------##
Usage/Quantity
non_Null    344509
dtype: int64
## ------------------------------------------##
Usage Units
non_Null    344509
dtype: int64
## ------------------------------------------##
Cost ($)
non_Null    331567
Null         12942
dtype: int64
\end{verbatim}


\subsubsection{Ranges}
From the range checking, one can see there are missing values for 'End
Date' (marked as 'inf') and illegal values for the 'Usage/Quantity'
(negative values)

\begin{verbatim}
        Portfolio Manager ID                       600                   4428021
  Portfolio Manager Meter ID                       519                  15550834
                    End Date                       inf       2015-09-01 00:00:00
              Usage/Quantity                -1385200.0               513798464.0
                    Cost ($)                       0.0                 7858632.0
\end{verbatim}

\note when pandas read in date time, it converts missing datetime data
to current date by default, which is why it shows up as inf (infinity)

\subsubsection{Non-negativity}
Checking the number of negative records for each group of energy
consumption. From the result, we can see there are 108 records in
District Chilled Water and 151 records in District Hot Water with
negative energy consumption records, which is identified as illegal
records that needs to be removed.

\begin{verbatim}
value  Meter Type                                     
<0     District Chilled Water - Electric                     108
       District Hot Water                                    151
>=0    District Chilled Water - Absorption                   153
       District Chilled Water - Electric                     528
       District Chilled Water - Engine                        49
       District Chilled Water - Other                       5925
       District Hot Water                                    220
       District Steam                                      15784
       Electric - Grid                                    111065
       Electric - Solar                                     1900
       Electric - Wind                                        12
       Fuel Oil (No. 2)                                    20534
       Natural Gas                                         79492
       Other Indoor                                           14
       Other:                                                580
       Other: Mixed Indoor/Outdoor                           467
       Potable Indoor                                       1177
       Potable: Mixed Indoor/Outdoor                       98474
       Power Distribution Unit (PDU) Input Meter              16
       Power Distribution Unit (PDU) Output Meter            106
       Uninterruptible Power Supply (UPS) Output Meter      7754
dtype: int64
\end{verbatim}

After removing 39 missing ``End Date'' and 259 negative
``Usage/Quantity'', there are 344211 legal records to be further
processed.

\subsection{Input data quality check - Static info}
\subsubsection{Missing data in PM file}
There are no missing data for the static information in sheet-0 of the PM file
\begin{verbatim}
checking numer of missing values for columns
## ------------------------------------------##
Property Name
non_Null    850
dtype: int64
## ------------------------------------------##
Portfolio Manager ID
non_Null    850
dtype: int64
## ------------------------------------------##
State/Province
non_Null    850
dtype: int64
## ------------------------------------------##
Postal Code
non_Null    850
dtype: int64
## ------------------------------------------##
Year Built
non_Null    850
dtype: int64
## ------------------------------------------##
Gross Floor Area
non_Null    850
dtype: int64
\end{verbatim}

\subsubsection{Duplicate and missing record in EUAM template}
First read in ``Building ID'' and ``Region'' columns from EUAM
template table, for each (``Building ID'', ``Region'')pair, there are
11 duplicate records.

\begin{verbatim}
# number of records
Building ID    11713 non-null object
Region         11713 non-null int64

# number of unique values
Building ID: 1065
Region: 11
\end{verbatim}

Per Jiyhun's advice, I should look up the 'Region' field with building
id in PM table from the EUAS template, since there are more buildings
in EUAS (1065) than in PM(850). However, after reading both tables, I
found there are only 120 common buildings in the two files, which
means one cannot use EUAS table as a lookup table to retrieve region
information for buildings in the PM file.

\begin{verbatim}
850 buildings in PM
1065 buildings in EUAS
120 common building records between PM and EUAS
\end{verbatim}

Jiyhun pointed out the link to the definition of the
\href{http://www.gsa.gov/portal/category/22227}{GSA lookup}, and
\href{http://www.gsa.gov/graphics/pbs/RWAmap.pdf}{GSA region map}. 

There is a Canada state from the PM file, the following output shows
the number of buildings in each state/Country in the PM file:
\begin{verbatim}
Canada         British Columbia                1
United States  Alabama                        12
               Alaska                         10
               Arizona                        16
               Arkansas                        9
               California                     44
               Colorado                       43
               Connecticut                     6
               Delaware                        1
               District of Columbia (D.C.)    44
               Florida                        25
               Georgia                        27
               Hawaii                          3
               Idaho                           4
               Illinois                       23
               Indiana                        10
               Iowa                            6
               Kansas                          4
               Kentucky                       11
               Louisiana                      14
               Maine                          24
               Maryland                       35
               Massachusetts                  16
               Michigan                       19
               Minnesota                      14
               Mississippi                     9
               Missouri                       11
               Montana                        10
               Nebraska                        5
               Nevada                          6
               New Hampshire                   5
               New Jersey                     11
               New Mexico                     14
               New York                       48
               North Carolina                 14
               North Dakota                   15
               Ohio                           21
               Oklahoma                        7
               Oregon                         11
               Pennsylvania                   15
               Puerto Rico                     4
               Rhode Island                    2
               South Carolina                 14
               South Dakota                    5
               Tennessee                      12
               Texas                          80
               Utah                            9
               Vermont                        24
               Virgin Islands of the U.S.      2
               Virginia                       17
               Washington                     34
               West Virginia                  12
               Wisconsin                       6
               Wyoming                         6
\end{verbatim}

The non-U.S. state should be dropped in the analysis, because it is
not in the definition of GSA region. 

\pagebreak
\subsection{Clean up summary}
\begin{itemize}
\item PM sheet-0, static information
  \begin{itemize}
  \item Turn `Property Name' from `XXXXXXXX - XXXXXXXXXX' to just the
    string before `-'
  \item Keep only 5 digit for zip code
  \item Drop non-U.S. state
  \end{itemize}
\item PM sheet-5, energy information
  \begin{itemize}
  \item Drop missing data for `End Date'
  \item Mark missing cost data as `-1'
  \item Drop negative energy consumption record
  \item Drop non-U.S. state
  \end{itemize}
The cleaned up energy information
\begin{verbatim}
Checking non-negativity after initial clean
>=0    344211
dtype: int64
is_nn  Meter Type                                     
>=0    District Chilled Water - Absorption                   153
       District Chilled Water - Electric                     528
       District Chilled Water - Engine                        49
       District Chilled Water - Other                       5925
       District Hot Water                                    220
       District Steam                                      15784
       Electric - Grid                                    111065
       Electric - Solar                                     1900
       Electric - Wind                                        12
       Fuel Oil (No. 2)                                    20495
       Natural Gas                                         79492
       Other Indoor                                           14
       Other:                                                580
       Other: Mixed Indoor/Outdoor                           467
       Potable Indoor                                       1177
       Potable: Mixed Indoor/Outdoor                       98474
       Power Distribution Unit (PDU) Input Meter              16
       Power Distribution Unit (PDU) Output Meter            106
       Uninterruptible Power Supply (UPS) Output Meter      7754

# Ranges of columns
        Portfolio Manager ID                       600                   4428021
  Portfolio Manager Meter ID                       519                  15550834
                    End Date       1998-06-30 00:00:00       2015-09-01 00:00:00
              Usage/Quantity                       0.0               513798464.0
                    Cost ($)                      -1.0                 7858632.0

\end{verbatim}

\end{itemize}

\subsubsection{Retrieving region information summary}
I digitized the GSA region in into a table from
\href{http://www.gsa.gov/portal/category/22227}{GSA lookup} and
combined it with the data frame with the static information. State
abbreviation table is retrieved from
\url{http://www.stateabbreviations.us/}

Attention should be paid for U.S. owned island, since they don't
appear in \href{http://www.gsa.gov/graphics/pbs/RWAmap.pdf}{GSA region
  map}.
\section{Cleaned up and combined file}
After cleaning the static and the energy information, retrieving
region information and merging these three tables: static, energy and
region, there are 344118 records in the cleaned up table.
\begin{verbatim}
Data columns (total 16 columns):
Portfolio Manager ID          344118 non-null int64
Portfolio Manager Meter ID    344118 non-null int64
Meter Type                    344118 non-null object
End Date                      344118 non-null object
Usage/Quantity                344118 non-null float64
Usage Units                   344118 non-null object
Cost ($)                      344118 non-null float64
Year                          344118 non-null int64
Month                         344118 non-null int64
Building ID                   344118 non-null object
State                         344118 non-null object
Postal Code                   344118 non-null int64
Country                       344118 non-null object
Year Built                    344118 non-null int64
GSF                           344118 non-null int64
Region                        344118 non-null int64
\end{verbatim}
\newpage
\bibliographystyle{plain} \bibliography{myCitation}
\end{document}